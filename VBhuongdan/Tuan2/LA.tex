\subsection{Giới thiệu về ma trận}
Ta xét bảng số sau:
\[ \mathbf{A}=
\begin{bmatrix}
    1&5&12\\
    3&0&4\\
    0&7&9
\end{bmatrix}.
\]
Đây là một ô vuông có kích thước \(3\times 3\), tức là có \(3\) \emph{hàng} và \(3\) \emph{cột}. Hàng được đọc từ trên xuống và cột được đọc từ trái sang. Mỗi một phần tử  trong \(9\) phần tử  của bảng số này được xác định với một cặp số duy nhất của hàng và cột. Ví dụ, số \(4\) nằm ở hàng thứ hai và cột thứ ba. 
Các số \(12,4,9\) đều nằm ở cột thứ ba và các số \(3,0,4\) đều nằm ở hàng thứ hai. 

Hay ta cũng có thể lấy thêm một bảng số khác, chẳng hạn
\[\mathbf{B}=\begin{bmatrix}
    -1.3&0.6\\
    20.4&5.5\\
    9.7&-6.2
\end{bmatrix}\] Đây là một bảng số với \(3\) hàng và \(2\) cột. Nếu vẫn giữ nguyên cách đọc bảng số trước đó, thì số \(9.7\) có vị trí là hàng thứ ba, cột thứ nhất. 

Vậy ý nghĩa của những bảng số (ma trận) vừa rồi là gì? Ta hãy cùng xem xét thêm một ví dụ: hai bảng số có \(3\) hàng và \(1\) cột, 
\[\mathbf{v}=\begin{bmatrix}
    a\\b\\c
\end{bmatrix}, \quad \mathbf{w}=\begin{bmatrix}
    c\\d\\f
\end{bmatrix}.
\] Điều đáng chú ý ở đây là ta có thể gọi \(\mathbf{v}\) và \(\mathbf{w}\) là các \emph{vector}. Thật vậy, nếu ta để chúng tuân theo các quy tắc của vector, các thành phần của hai bảng số vừa rồi sẽ giống như là các thành phần của một vector. 
Nghĩa là, \[\mathbf{v}+\mathbf{w}=\begin{bmatrix}
    a+c\\b+d\\c+f
\end{bmatrix},\] hay \[
    4\cdot\mathbf{v}=\begin{bmatrix}
        4a\\4b\\4c
\end{bmatrix}.\]
Về cơ bản, đây chỉ là một sự thay đổi về cách viết. Cụ thể là thay vì viết \((a,b,c)\), ta viết \(\begin{bmatrix}
    a\\b\\c
\end{bmatrix}\). Như vậy chuyện gì xảy ra với  \(\mathbf{A}\) và \(\mathbf{B} \)? Chúng cũng là các vector (theo nghĩa trừu tượng hơn), nhưng tạm thời ta có thể chỉ cần nhìn nhận theo khía cạnh: \emph{các cột của chúng là các vector. }
\vspace{8pt}

    \emph{Ma trận là một mảng chữ nhật hoặc hình vuông (ma trận vuông) chứa các số hoặc những đối tượng toán học khác, mà có thể định nghĩa một số phép toán như cộng hoặc nhân trên các ma trận.}
\vspace{8pt}

Một ma trận \(\mathbf{A}\) có \(m\) hàng và \(n\) cột được gọi là một ma trận \(m\times n\), điều này xác dịnh độ lớn của ma trận. Ta viết \(\mathbf{A}_{m\times n}\) để chỉ ma trận \(A\) có kích thước \(m\times n\). Chú ý rằng ta đọc hàng trước cột. 


\subsection{Các phép toán trên ma trận}
Cũng như với các số và vector, ta có thể thực hiện pháp cộng, trừ với các ma trận, cũng có thể nhân một số với ma trận và cuối cùng là nhân ma trận với ma trận, ma trận với vector. 
\vspace{8pt}

\textbf{Phép cộng hai ma trận.} Xét hai ma trận \(\mathbf{A}\) và \(\mathbf{B}\) có kích thước \(m\times n\), tổng của hai ma trận là một ma trận \(m\times n\) được định nghĩa là 
\begin{equation}
   \mathbf{A}_{ij}\in\mathbb{R},\quad \mathbf{B}_{ij}\in\mathbb{R},\quad 
    (\mathbf{A}+\mathbf{B})_{ij}=\mathbf{A}_{ij}+\mathbf{B}_{ij}.
\end{equation}
Chú ý rằng ta viết \(\mathbf{A}_{ij}\) để chỉ phần tử nằm ở hàng thứ \(i\) và cột thứ \(j\) của \(\mathbf{A}\), và tương tự \((\mathbf{A+B})_{ij}\) để chỉ phần tử nằm ở hàng thứ \(i\) và cột thứ \(j\) của ma trận đó. Vậy, để cộng hai ma trận, ta cộng từng phần tử lại với nhau. Điều này tương tự như phép cộng các vector.
Tương tự, chúng ta có thể nhân ma trận với một hằng số \(c\in\mathbb{R}\):
\begin{equation}
    (c\mathbf{A})_{ij}=c\mathbf{A}_{ij}.
\end{equation}
Điều này tương tự như phép nhân vô hướng với một vector. Khi \(c=-1\), ta thu được ma trận \(-\mathbf{A}\) sao cho \(\mathbf{A}+(-\mathbf{A})=\mathbf{0}\); \(\mathbf{0}\) là ma trận kích thước \(m\times n\) với mọi phần tử trong đó đều bằng \(0\).
\vspace{8pt}

\textbf{Phép nhân ma trận-vector.} Ta hãy bắt đầu với một ví dụ. Giả sử ta có vector \[\mathbf{v}=\begin{bmatrix}
    +2\\+5\\-4
\end{bmatrix},\] vector này có thể được phân tích thành một tổ hợp tuyến tính của một hệ cơ sở nào đó, chẳng hạn
\[\begin{bmatrix}
    2\\5\\-4
\end{bmatrix}=2\begin{bmatrix}
    1\\0\\0
\end{bmatrix}+(-3)\begin{bmatrix}
    0\\2\\-5
\end{bmatrix}+1\begin{bmatrix}
    0\\11\\-19
\end{bmatrix}.\]
Bằng cách định nghĩa một phép toán mới, ta có thể viết lại thành dạng
\begin{equation}\begin{bmatrix}
    2\\5\\-4
\end{bmatrix}=\begin{bmatrix}
    1&0&0\\
    0&2&11\\
    0&-5&-19
\end{bmatrix}\begin{bmatrix}
    2\\-3\\1
\end{bmatrix}.\end{equation}\label{eq2.2.1} Ta đã đặt các vector cơ sở vào cột của ma trận \(3\times 3\) vừa rồi, và các hệ số của tổ hợp tuyến tính vào một vector cột. Về tổng quát, một phép nhân ma trận \(m\times n\) với một vector \(n\times 1\) sẽ cho ra một vector \(m\times 1\), và phần tử thứ \(i\) được tính bởi 
\begin{equation}
    (\mathbf{Ax})_i = \sum_{j=1}^n \mathbf{A}_{ij}\mathbf{x}_j.
\end{equation}
Vector mới là một tổ hợp tuyến tính của các cột của ma trận \(\mathbf{A}\) với các hệ số là các phần tử của vector \(\mathbf{x}\). Cũng dễ thấy rằng, phần tử thứ \(i\) của vector này là tích vô hướng của hàng thứ \(i\) của \(\mathbf{A}\) với vector \(\mathbf{x}\). Nghĩa là, chẳng hạn, 
\[\begin{bmatrix}
    0&2&11
\end{bmatrix}\begin{bmatrix}
    2\\-3\\1
\end{bmatrix}=\begin{bmatrix}
    0\\2\\11
\end{bmatrix}\cdot \begin{bmatrix}
    2\\-3\\1
\end{bmatrix} =5.\]
Vì tích vô hướng có tính phân phối là \(\mathbf{a}\cdot(\mathbf{b}+\mathbf{c})=\mathbf{a}\cdot\mathbf{b}+\mathbf{a}\cdot\mathbf{c},\)
phép nhân ma trận-vector cũng có tính chất tương tự: \[\mathbf{A}(\mathbf{a}+\mathbf{b})=\mathbf{A}\mathbf{a}+\mathbf{A}\mathbf{b}.\]
\vspace{8pt}

\textbf{Phép nhân ma trận với ma trận.} Ta bắt đầu với việc biểu diễn các vector cơ sở được nhắc tới vừa rồi thông qua một hệ cơ sở khác.
\begin{align*}&\begin{bmatrix}
    1\\0\\0
\end{bmatrix}=1\begin{bmatrix}
    0.5 \\-1\\0
\end{bmatrix}+2\begin{bmatrix}
    0.75 \\1\\-2
\end{bmatrix}+1\begin{bmatrix}
    -1 \\-1\\4
\end{bmatrix},\\
&\begin{bmatrix}
    0\\2\\-5
\end{bmatrix}=-1.625\begin{bmatrix}
    0.5\\-1\\0
\end{bmatrix}-1.75\begin{bmatrix}
    0.75\\-1\\0
\end{bmatrix}-2.125\begin{bmatrix}
    -1\\-1\\4
\end{bmatrix},\\
&\begin{bmatrix}
    0\\11\\-19
\end{bmatrix}=-7.875\begin{bmatrix}
    0.5\\-1\\0
\end{bmatrix}-3.25\begin{bmatrix}
    0.75\\-1\\0
\end{bmatrix}-6.375\begin{bmatrix}
    -1\\-1\\4
\end{bmatrix}.
\end{align*}
Các tổng này, như đã biết, có thể được viết thành tích của một ma trận và một vector:
\begin{align*}
    &\begin{bmatrix}
        1\\0\\0
    \end{bmatrix}=\begin{bmatrix}
        0.5 & 0.75 & -1\\
        -1 & 1 & -1\\
        0 & -2 & 4
    \end{bmatrix}\begin{bmatrix} 1\\2\\1\end{bmatrix},\\
&\begin{bmatrix}
    0\\2\\-5
\end{bmatrix}=\begin{bmatrix}
    0.5 & 0.75 & -1\\
    -1 & 1 & -1\\
    0 & -2 & 4
\end{bmatrix}\begin{bmatrix}
    -1.625\\-1.75\\-2.125
\end{bmatrix}\\
&\begin{bmatrix}
    0\\11\\-19
\end{bmatrix}=\begin{bmatrix}
    0.5 & 0.75 & -1\\
    -1 & 1 & -1\\
    0 & -2 & 4
\end{bmatrix}\begin{bmatrix}
    -7.875\\-3.25\\-6.375
\end{bmatrix}.
\end{align*} Gọi ma trận \(3\times 3\) ở vế phải là \(\mathbf{B}\), thay vào \eqref{eq2.2.1},
\[\begin{bmatrix}
    2\\5\\-4
\end{bmatrix}=\begin{bmatrix}
    \mathbf{B}\begin{bmatrix}
        1\\2\\1
    \end{bmatrix}&\mathbf{B}\begin{bmatrix}
        -1.625\\-1.75\\-2.125
    \end{bmatrix}&\mathbf{B}\begin{bmatrix}
        -7.875\\-3.25\\-6.375
\end{bmatrix}\end{bmatrix}\begin{bmatrix}
2\\-3\\1
\end{bmatrix}.\] Quan sát phương trình này, ta nhận thấy sự lặp của \(\mathbf{B}\); điều này liên tưởng ta đến một phép nhân. 
Tức là, ta có thể viết 
\[\begin{bmatrix}
    2\\5\\-4
\end{bmatrix}=\mathbf{B}\begin{bmatrix}
    1&-1.625&-7.875\\
    2&-1.75&-3.25\\
    1&-2.125&-6.375
\end{bmatrix}\begin{bmatrix}
    2\\-3\\1
\end{bmatrix},\] bằng cách định nghĩa một phép nhân mới, và ta gọi phép nhân này là một phép nhân ma trận với ma trận.
\vspace{8pt}

Xét một ma trận \(\mathbf{A}_{m\times n}\) và một ma trận \(\mathbf{C}_{n\times p}\), \emph{tích của của chúng là một ma trận \(\mathbf{C}_{m\times p}\); các cột của ma trận này là các vector, bằng với tích ma trận-vector của ma trận \(\mathbf{A}\) và các cột tương ứng  của ma trận \(B\).}
\vspace{8pt}

Đồng thời, ta cũng nhận thấy rằng tích ma trận-vector cũng là một tích ma trận-ma trận, vì vector là một ma trận có một cột. Do đó, để tổng quát, ta định nghĩa phép nhân ma trận với ma trận như sau: 
\begin{equation}
    \mathbf{C}_{ij}=\mathbf{AB}_{ij}=\sum_{k=1}^n \mathbf{A}_{ik}\mathbf{B}_{kj}.
\end{equation}\label{eq2.2.2}
Hay, nói cách khác, phần tử thứ \((i,j)\) của \(\mathbf{C}\) bằng tích vô hướng của hàng thứ \(i\) của ma trận \(\mathbf{A}\) với cột thứ \(j\) của ma trận \(\mathbf{B}\).
\vspace{8pt}

Để kết thúc phần này, ta hãy xét thêm một ví dụ. Hãy tính tích
\[\begin{bmatrix}
    1&5\\ 3&2
\end{bmatrix}\begin{bmatrix}
    2&-1\\ 0&3
\end{bmatrix}\] theo hai cách: \eqref{eq2.2.2} và bằng góc nhìn của phép nhân vector.

\emph{Giải.} Theo \eqref{eq2.2.2}, tích này bằng 
\[\begin{bmatrix}
    (1\cdot 2+5\cdot 0)&(1\cdot -1+5\cdot 3)\\
   ( 3\cdot 2+2\cdot 0)&(3\cdot -1+2\cdot 3)
\end{bmatrix}=\begin{bmatrix}
    2&14\\
    6&3
\end{bmatrix}.\]
Theo góc nhìn của phép nhân vector, tích này tương đương với 
\[\begin{bmatrix}
    \begin{bmatrix}
        1&5\\3&2
    \end{bmatrix}\begin{bmatrix}
        2\\0
    \end{bmatrix} &\begin{bmatrix}
        1&5\\3&2
    \end{bmatrix}\begin{bmatrix}
        -1\\3
    \end{bmatrix}
\end{bmatrix}.\]
Dễ thấy, \begin{align*}
    &\begin{bmatrix}
        1&5\\3&2
    \end{bmatrix}\begin{bmatrix}
        2\\0
    \end{bmatrix}=2\begin{bmatrix}
        1\\3
    \end{bmatrix}+0\begin{bmatrix}
        5\\2
    \end{bmatrix}=\begin{bmatrix}
        2\\6
    \end{bmatrix},\\
    &\begin{bmatrix}
        1&5\\3&2
    \end{bmatrix}\begin{bmatrix}
        -1\\3
    \end{bmatrix}=-1\begin{bmatrix}
        1\\3
    \end{bmatrix}+3\begin{bmatrix}
        5\\2
    \end{bmatrix}=\begin{bmatrix}
        14\\3
    \end{bmatrix}.
\end{align*}

\subsection{Phép biến đổi tuyến tính}
