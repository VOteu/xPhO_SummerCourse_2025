\titleformat{\chapter}[display]
  {\normalfont\Large\bfseries}{\centering Tuần 9}{10pt}{\centering\Huge\bfseries}
  
\chapter{Nhập Môn Cơ Học Giải Tích}

\section{Nguyên lý tác dụng tối thiểu}

\subsection{Nguyên lý biến phân}

\subsection{Phương trình Euler-Lagrange}


\section{Cơ học Lagrange}

\subsection{Phương trình Lagrange loại II}

\subsection{Phương trình Lagrange loại I}

\subsection{Động lượng suy rộng}

\subsection{Giải phương trình chuyển động bằng phương pháp Runge-Kutta 4}

\subsection{Tính toán lực bị động dựa trên phương trình Lagrange loại 2}


\section{Định lý Noether}

% Continuous symmetry

% Bài tập ví dụ: Xác định phương trình vi phân mô tả chuyển động của con lắc kép.

\section{Các lý thuyết cơ học giải tích khác}

\subsection{Cơ học Hamilton}

\subsection{Nguyên lý Gauss về liên kết tối thiểu}

\subsection{Phương trình Appell cho cơ hệ phi Holonom}


\section{Bài tập}

\section{Lời giải}