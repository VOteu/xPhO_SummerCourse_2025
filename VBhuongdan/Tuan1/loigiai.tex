\textbf{Bài 1.1:}
\begin{enumerate}[label=(\alph*)]
    \item Hàm số xác định nếu \(2x-1\neq 0\), hay \(x\neq \frac{1}{2}\). Vì vậy \(X=(-\infty, 1/2)\cup (1/2,\infty)\).
    \item Hàm này xác định nếu \(x-1\neq 0\) và \(1+x>0\), hay \(x\neq 1\) và \(x>-1\). Vì vậy \(X=(-1,1)\cup (1,\infty)\).
    \item Số hạng thứ nhất nhận các giá trị thực khi \(x\leq\frac{1}{2}\). Số hạng thứ hai nhận các giá trị thực khi \(-1\leq\frac{3x-1}{2}\leq 1.\) Giải ra ta được \(x\leq 1, x\geq -\frac{1}{3}\). Do đó miền xác định là đoạn \([-1/3,1/2]\).
    \item Hàm số xác định với \(x\neq 0\). Nên \(X=(-\infty, 0)\cup(0,\infty)\).
    \item Điều kiện để hàm số xác định là \(3x+1\geq 0\) và \(x+1\geq 0\). Vậy \(X=[-1/3,\infty)\).
\end{enumerate}
\vspace{5pt}

\textbf{Bài 1.2:}
\begin{enumerate}[label=(\alph*)]
    \item Biến đổi, ta được \(f(x)=(x-3)^2 -4 \geq -4\). Do đó tập hợp giá trị của hàm là khoảng \(Y=[-4, \infty)\).
    \item Vì \(-1\leq \sin x\leq 1\), nên \(-3\leq\sin x\leq 3\). Do đó \(-1\leq f(x)\leq 5\) và \(Y=[-1, 5]\).
    \item Ta xem xét hai trường hợp : \(x<0 \text{ và } x>0\). \newline Nếu \(x<0\), \(y+\lvert y\rvert =1\). Giá trị của \(y\) không thể nhỏ hơn \(0\) vì điều này tương đương với \(y+\lvert y\rvert =0\). Do đó \(y=\frac{1}{2}\) trong trường hợp này.\newline Nếu \(x>0\), \(y\) chắc chắn lớn hơn \(0\) và do đó ta thu được hàm \(y=x+\frac{1}{2}\).\newline Dễ thấy, miền giá trị \(Y=[1/2,\infty)\).
    \item Miền giá trị của hàm là \(Y=[0,4]\).
\end{enumerate}
\vspace{5pt}

\textbf{Bài 1.3:}

Xét đường tròn tâm \(O\) có đường kính \(AC\) với độ dài bằng \(1\) và tứ giác \(ABCD\) nội tiếp đường tròn như hình bên.\raisebox{-0.2em}{\includegraphics[height=8 cm]{Tuan1/ảnh/unitcircle.png}}\newline
Vì \(\lvert AC\rvert =1\) nên tất cả độ dài trong biểu đồ được kết nối với sine hoặc cosine, và chú ý rằng hai góc ở đỉnh \(B\) và \(D\) là các góc vuông.\newline
Lúc này, theo định lý hàm sine trong tam giác, độ dài đoạn \(BD\) chính là \(\sin(\alpha+\beta)\). Định lý Ptoleme lại cho biết rằng \[\lvert AC\rvert \cdot\lvert BD\rvert =\lvert AB\rvert\dot\lvert CD\rvert +\lvert BC\rvert\dot\lvert AD\rvert.\]
và biểu thức này trở thành, sau khi ta thay các giá trị sine/cosine tương ứng:\[\sin(\alpha+\beta)=\sin\alpha\cos\beta +\sin\beta\cos\alpha.~(Q.E.D)\]
Sử dụng định lý Pythagoras, ta chứng minh được đẳng thức còn lại.

\vspace{5pt}
\textbf{Bài 1.10:}
\begin{enumerate}[label=(\alph*)]   
    \item Hàm số liên tục tại \(x=4\). Vậy nên \[\lim_{x\rightarrow 4}\frac{5x+2}{2x+3}=\frac{5\times 4+2}{2\times 4+3}=2.\]
    \item Ta thấy hàm số có dạng vô định \(\frac{\infty}{\infty}\) khi \(x\rightarrow\infty\). Một phương pháp điển hình để giải quyết tình huống này đó là chia bậc lớn nhất của \(x\) cho cả tử và mẫu, mà ở đây là bậc một. Khi đó, \[\lim_{x\rightarrow\infty}\frac{3x+5}{2x+7}=\lim_{x\rightarrow\infty}\frac{3+\frac{5}{x}}{2+\frac{7}{x}}=\frac{3}{2}.\] 
    \item Giới hạn của hàm số này cũng có dạng vô định \(\frac{\infty}{\infty}\). Tương tự, ta chia bậc lớn nhất của \(x\) cho cả tử và mẫu, mà ở đây là bậc ba. Khi đó, \[\lim_{x\rightarrow\infty}\frac{x^3 +2x^2 +3x+4}{4x^3 +3x^2 +2x+1}=\lim_{x\rightarrow\infty}\frac{1+\frac{2}{x}+\frac{3}{x^2}+\frac{4}{x^3}}{4+\frac{3}{x}+\frac{2}{x^2}+\frac{1}{x^3}}=\frac{1}{4}.\]
    \item Ta cũng có dạng vô định \(\frac{\infty}{\infty}\). Chia bậc lớn nhất  \(x^4\) cho cả tử và mẫu . Khi đó, \[\lim_{x\rightarrow\infty}\frac{3x^4 -2}{\sqrt{x^8+3x+4}}=\lim_{x\rightarrow\infty}\frac{3-\frac{2}{x^4}}{\sqrt{1+\frac{3}{x^7}+\frac{4}{x^8}}}=\frac{3}{1}=3.\]
    \item Ta có thể viết lại biểu thức này như sau: \[\lim_{x\rightarrow\infty}\sqrt{x^2 +8x+3}-\sqrt{x^2+4x+3}=\lim_{x\rightarrow\infty}\frac{(x^2 +8x+3)-(x^2+4x+3)}{\sqrt{x^2 +8x+3}+\sqrt{x^2+4x+3}}.\] Khi đó, chia bậc lớn nhất \(x\) cho cả tử và mẫu, và thu được, \[\lim_{x\rightarrow\infty}\frac{4}{\sqrt{1+\frac{8}{x}+\frac{3}{x^2}}+\sqrt{1+\frac{4}{x}+\frac{3}{x^2}}}=\lim_{x\rightarrow\infty}\frac{4}{2}=2.\]
    \item Giới hạn của hàm số này, nếu thay trực tiếp \(x=3\) sẽ có dạng vô định \(\frac{0}{0}\). Một phương pháp điển hình để giải quyết tình huống này, nếu có thể, là khử đi các nhân tử chung trong tử và mẫu. Dễ thấy, \[\lim_{x\rightarrow 3}\frac{x^2 -9}{x^2-3x}=\lim_{x\rightarrow 3}\frac{(x-3)(x+3)}{(x-3)x}=\lim_{x\rightarrow 3}\frac{x+3}{x}=6.\]  
    \item Giới hạn của hàm số này, nếu thay trực tiếp, cũng có dạng vô định \(\frac{0}{0}\). Triển khai tương tự, ta thu được \[\lim_{x\rightarrow 1}\frac{x^3 -x^2 -x+1}{x^3+x^2 -x-1}=\lim_{x\rightarrow 1}\frac{(x-1)(x^2+x+1)}{(x-1)(x^2+2)}=\lim_{x\rightarrow 1}\frac{x^2+x+1}{x^2+2}=1.\]
    \item Để khử nhân tử chung, trước tiên, nhân liên hợp tử và mẫu với \(\sqrt{1+x+x^2}+\sqrt{7+2x-x^2}\). Như vậy, ta có thể viết lại giới hạn này như sau: \[\lim_{x\rightarrow 2}\frac{(1+x+x^2)-(7+2x-x^2)}{(x^2-2x)(\sqrt{1+x+x^2}+\sqrt{7+2x-x^2})}.\] Kết quả thu được là \[\lim_{x\rightarrow 2}\frac{(2x+3)(x-2)}{x(x-2)(\sqrt{1+x+x^2}+\sqrt{7+2x-x^2})}=\frac{\sqrt{7}}{4}.\]    
\end{enumerate}
\vspace{5pt}

\textbf{Bài 1.11:}
\begin{enumerate}[label=(\alph*)]
\item \[\lim_{x\rightarrow 0}\frac{\sin mx}{x}=m\lim_{x\rightarrow 0}\frac{\sin mx}{mx}=m.\]
\item Sử dụng kết quả của \textbf{Bài 1.5}, ta có \[\cos 5x =\cos^2 \left(\frac{5x}{2}\right)-\sin^2\left(\frac{5x}{2}\right)=1-2\sin^2\left(\frac{5x}{2}\right).\] Thay vào giới hạn, kết quả thu được là \[\lim_{x\rightarrow 0}\frac{2\sin^2\left(\frac{5x}{2}\right)}{x^2}=2\times\frac{25}{4}\lim_{x\rightarrow 0}\frac{\sin^2 \left(\frac{5x}{2}\right)}{\left(\frac{5x}{2}\right)^2}=\frac{25}{2}.\]
\item Hướng giải quyết ở đây sẽ là đưa giới hạn hàm số này về giới hạn đáng nhớ (b): 
\begin{align*}
    \lim_{x\rightarrow\infty}\left(\frac{x^2+5x+4}{x^2-3x+7}\right)^x&=\lim_{x\rightarrow\infty}\left(1+\frac{8x-3}{x^2 -3x+7}\right)^{\frac{x^2-3x+7}{8x-3}\cdot\frac{8x^2-3x}{x^2-3x+7}}\\ &=
\lim_{x\rightarrow\infty}\left(\left(1+\frac{8x-3}{x^2 -3x+7}\right)^{\frac{x^2-3x+7}{8x-3}}\right)^{\frac{8x^2-3x}{x^2-3x+7}}.
\end{align*}
 Dễ thấy, \[\lim_{x\rightarrow\infty}\frac{8x^2-3x}{x^2-3x+7}=8,\] và \[\lim_{x\rightarrow\infty}\frac{x^2-3x+7}{8x-3}=\infty.\] Thành thử, kết quả thu được là \[\lim_{x\rightarrow\infty}\left(\frac{x^2+5x+4}{x^2-3x+7}\right)^x=e^8.\] 
\item Tương tự, kết quả sẽ thu được là \(\sqrt{e}\).
\end{enumerate}
\vspace{5pt}

\textbf{Bài 1.12:}  
\begin{enumerate}[label=(\alph*)]   
    \item \(\frac{1}{3}\)
    \item \(-\frac{1}{2}\)
    \item \(\frac{9}{25}\)
\end{enumerate}
\vspace{5pt}

\textbf{Bài 1.14:}
\begin{enumerate}[label=(\alph*)]
\item \[y' =\frac{1}{\sqrt{x^2+1}}.\]
\item \[y'= \frac{\cos x}{\sqrt{4\sin^2 x -1}}.\]
\item \[y'= \sqrt{x^2+k}.\]
\item \[y'=\frac{\tan^2 x+1}{\sqrt{\tan x (4\tan x +1)}}\ln \left(8\tan x +4\sqrt{\tan x (4\tan x +1)}+1\right).\]
\item \[y'=\sqrt{x^2 +2\alpha x +\beta}.\]
\item \[y'=\frac{4\sqrt{x\left(x+\sqrt{x}\right)}+2\sqrt{x}+1}{8\sqrt{\left(x+\sqrt{x+\sqrt{x}}\right)(x+\sqrt{x})x}}.\]
\end{enumerate}  