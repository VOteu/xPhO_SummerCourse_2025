\titleformat{\chapter}[display]
  {\normalfont\Large\bfseries}{\centering Tuần 4}{10pt}{\centering\Huge\bfseries}
  
\chapter{Cơ Động Lực Học Chất Điểm}
Động học mô tả sự chuyển động của vật nhưng chưa đề cập đến nguồn cơn của những chuyển động đó.
Động lực học nghiên cứu sự chuyển động của vật liên hệ với các nguyên nhân (các tương tác giữa các vật) gây ra một đặc trưng nào đó của chuyển động.
\vspace{8pt}

Ba định luật động lực học được Newton trình bày vào năm 1687 dựa trên sự tổng kết một loạt các kết quả thực nghiệm 
tuy còn nhiều hạn chế nhưng đã thành công trong ứng dụng ở một phạm vi rất lớn các hiện tượng quen thuộc trong đời sống
-những vật thể lớn hơn nhiều so với kích thước của các nguyên tử và có vận tốc nhỏ hơn nhiều so với vận tốc ánh sáng.
Đây chính là cơ sở của cơ học cổ điển.


\section{Ba Định Luật Newton}
\subsection{Định luật thứ nhất}
\subsection{Định luật thứ hai}
\subsection{Định luật thứ ba}
\subsection{Một số "loại" động lượng khác}

\section{Nguyên lý tương đối Galileo}
\subsection{Phép biến đổi Galileo}
\subsection{Luận bàn}

\section{Các lực cơ học}
\input{Tuan4/forces.tex}
\section{Liên kết}
\subsection{Các ràng buộc hình học}
\subsection{Vai trò của các loại lực liên kết}

\section{Phương pháp tiếp cận một bài toán động lực học}