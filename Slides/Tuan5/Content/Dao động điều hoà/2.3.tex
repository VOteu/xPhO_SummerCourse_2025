\begin{frame}{Hệ DĐ cưỡng bức}
    \begin{center}
        \resizebox{0.7\linewidth}{!}{


% Pattern Info
 
\tikzset{
pattern size/.store in=\mcSize, 
pattern size = 5pt,
pattern thickness/.store in=\mcThickness, 
pattern thickness = 0.3pt,
pattern radius/.store in=\mcRadius, 
pattern radius = 1pt}
\makeatletter
\pgfutil@ifundefined{pgf@pattern@name@_o3npe040r}{
\pgfdeclarepatternformonly[\mcThickness,\mcSize]{_o3npe040r}
{\pgfqpoint{0pt}{0pt}}
{\pgfpoint{\mcSize+\mcThickness}{\mcSize+\mcThickness}}
{\pgfpoint{\mcSize}{\mcSize}}
{
\pgfsetcolor{\tikz@pattern@color}
\pgfsetlinewidth{\mcThickness}
\pgfpathmoveto{\pgfqpoint{0pt}{0pt}}
\pgfpathlineto{\pgfpoint{\mcSize+\mcThickness}{\mcSize+\mcThickness}}
\pgfusepath{stroke}
}}
\makeatother
\tikzset{every picture/.style={line width=0.75pt}} %set default line width to 0.75pt        

\begin{tikzpicture}[x=0.75pt,y=0.75pt,yscale=-1,xscale=1]
%uncomment if require: \path (0,15225); %set diagram left start at 0, and has height of 15225

%Straight Lines [id:da03890405315517542] 
\draw [color={rgb, 255:red, 0; green, 0; blue, 0 }  ,draw opacity=1 ]   (77.5,595.22) -- (78,680.22) ;
%Straight Lines [id:da6771080202521618] 
\draw [color={rgb, 255:red, 0; green, 0; blue, 0 }  ,draw opacity=1 ]   (303.5,680.22) -- (78,680.22) ;
%Shape: Rectangle [id:dp8957593748114105] 
\draw  [draw opacity=0][pattern=_o3npe040r,pattern size=6pt,pattern thickness=0.75pt,pattern radius=0pt, pattern color={rgb, 255:red, 0; green, 0; blue, 0}] (77.5,595.22) -- (52.5,595.22) -- (52.5,680.22) -- (77.5,680.22) -- cycle ;
%Shape: Resistor [id:dp9587285208573366] 
\draw  [color={rgb, 255:red, 0; green, 0; blue, 0 }  ,draw opacity=1 ] (77.75,621.72) -- (94.4,621.72) -- (98.1,612.84) -- (105.5,630.59) -- (112.9,612.84) -- (120.3,630.59) -- (127.7,612.84) -- (135.1,630.59) -- (142.5,612.84) -- (149.9,630.59) -- (153.6,621.72) -- (170.25,621.72) ;
%Shape: Square [id:dp46764462046986255] 
\draw  [color={rgb, 255:red, 0; green, 0; blue, 0 }  ,draw opacity=1 ] (170.5,595.22) -- (255.5,595.22) -- (255.5,680.22) -- (170.5,680.22) -- cycle ;
%Straight Lines [id:da1727736810609144] 
\draw [color={rgb, 255:red, 0; green, 0; blue, 0 }  ,draw opacity=1 ]   (322.5,710.22) -- (216.5,710.22) ;
\draw [shift={(216.5,710.22)}, rotate = 360] [color={rgb, 255:red, 0; green, 0; blue, 0 }  ,draw opacity=1 ][line width=0.75]    (0,5.59) -- (0,-5.59)   ;
\draw [shift={(325.5,710.22)}, rotate = 180] [fill={rgb, 255:red, 0; green, 0; blue, 0 }  ,fill opacity=1 ][line width=0.08]  [draw opacity=0] (8.93,-4.29) -- (0,0) -- (8.93,4.29) -- cycle    ;
%Straight Lines [id:da5324610265708711] 
\draw [color={rgb, 255:red, 0; green, 0; blue, 0 }  ,draw opacity=1 ]   (216.5,710.22) -- (107.5,710.22) ;
%Straight Lines [id:da24682222137677545] 
\draw [color={rgb, 255:red, 0; green, 0; blue, 0 }  ,draw opacity=1 ]   (78.5,668.22) -- (114.5,668.22) ;
%Straight Lines [id:da08809092762479498] 
\draw [color={rgb, 255:red, 0; green, 0; blue, 0 }  ,draw opacity=1 ]   (131.5,668.22) -- (169.5,668.22) ;
%Shape: Rectangle [id:dp729685852038932] 
\draw  [color={rgb, 255:red, 0; green, 0; blue, 0 }  ,draw opacity=1 ][fill={rgb, 255:red, 155; green, 155; blue, 155 }  ,fill opacity=1 ][line width=0.75]  (129.42,664.02) -- (135.5,664.02) -- (135.5,673.22) -- (129.42,673.22) -- cycle ;
%Straight Lines [id:da9203533268837498] 
\draw [color={rgb, 255:red, 0; green, 0; blue, 0 }  ,draw opacity=1 ][line width=0.75]    (114,660.62) -- (139.96,660.62) ;
%Straight Lines [id:da1854839912471089] 
\draw [color={rgb, 255:red, 0; green, 0; blue, 0 }  ,draw opacity=1 ][line width=0.75]    (114.51,660.22) -- (114.51,676.22) ;
%Straight Lines [id:da7739527238409212] 
\draw [color={rgb, 255:red, 0; green, 0; blue, 0 }  ,draw opacity=1 ][line width=0.75]    (114.51,675.82) -- (140.47,675.82) ;

%Straight Lines [id:da7046823744172879] 
\draw [line width=2.25][color={wsdred}  ,draw opacity=1 ]   (255,641) -- (294.5,641) ;
\draw [shift={(299.5,641)}, rotate = 180] [fill={wsdred}  ][line width=0.08]  [draw opacity=0] (14.29,-6.86) -- (0,0) -- (14.29,6.86) -- cycle    ;

% Text Node
\draw (119,585.22) node [anchor=north west][inner sep=0.75pt]   [align=left] {\textcolor{black}{k}};
% Text Node
\draw (213,637.72) node   [align=left] {M};
% Text Node
\draw (216.5,717) node [anchor=north] [inner sep=0.75pt]   [align=left] {\textcolor{black}{O}};
% Text Node
\draw (118,637.22) node [anchor=north west][inner sep=0.75pt]   [align=left] {\textcolor{black}{b}};
% Text Node
\draw (281,613) node [anchor=north west][inner sep=0.75pt]{\textcolor{wsdred}{$\mathbf{F_3} \ =\ F_{0} \ \cos( \si{\ohm}t + \phi) \ \mathbf{e_{x}}$}};


\end{tikzpicture}}
    \end{center}
    Lúc này hệ sẽ chịu thêm một lực cưỡng bức \(\mathbf{F_3}\). Tổng lực tác động lên vật
    \begin{equation}
        \displaystyle \mathbf{F} =  - k \mathbf{x} - b \mathbf{x'} + F_0 \cos{(\Omega t + \phi)} \mathbf{e_x}.
    \end{equation}
\end{frame}
\begin{frame}{Hệ DĐ cưỡng bức - PTVP}
    Lúc này ta sẽ đi giải phương trình vi phân
    \begin{equation}
        mx'' = -kx -bx' + F_0 \cos{(\Omega t + \phi)}
    \end{equation}
    Cụ thể, ta sẽ đi giải lần lượt nghiệm thuần nhất và nghiệm riêng.
    \begin{equation*}
        x = x_{0} + x_r.
    \end{equation*}
    Để giải nghiệm thuần nhất, ta đi giải phương trình vi phân sau
    \begin{equation*}
    x''_{tn} + {\displaystyle \frac{b}{m}} x'_{tn} + {\displaystyle \frac{k}{m}} x'_{tn} = 0.
    \end{equation*}
\end{frame}
\begin{frame}{Nghiệm riêng}
    Ta đặt
    \begin{equation*}
        x_r = A \cos{\left(\Omega t + \phi\right)} + B \sin{\left(\Omega t + \phi\right)}
    \end{equation*}
    Thế vào phương trình vi phân, ta đồng nhất \(\sin{()}\) và \(\cos{()}\) hai vế, ta có
    \begin{equation}
        \begin{array}{ccc}
        A &=& \displaystyle \frac{F_0}{m} \frac{\omega^2 - \Omega^2}{\left(\omega^2 - \Omega^2 \right)^2 + \left( c \Omega \right)^2}. 
        \\
        \\
        B &=& \displaystyle \frac{F_0}{m} \frac{c \Omega}{\left(\omega^2 - \Omega^2 \right)^2 + \left( c \Omega \right)^2}.
        \end{array}
    \end{equation}
\end{frame}