\begin{frame}
    \frametitle{Hàm số và Đồ thị}
    \begin{tcolorbox}[colback=blue!10!, colframe=blue!50!black, title=Định nghĩa]
        \textit{Hàm} $f$ là một quy tắc cho tương ứng mỗi phần tử x thuộc tập hợp $D$ với một và chỉ một phần tử , kí hiệu $f(x)$, thuộc một tập hợp $E$.
    \end{tcolorbox}
    \begin{itemize}
        \item $D$ được gọi là \textit{miền xác định} của hàm số.
        \item $E$ được gọi là \textit{miền giá trị} của hàm số.
    \end{itemize}
    (hình vẽ)
    \end{frame}
    
    \begin{frame}{Hàm số và Đồ thị}
        Cách phổ biến nhất để minh hoạt một hàm số là thông qua đồ thị của nó.
        \begin{tcolorbox}[colback=blue!10!, colframe=blue!50!black, title=]
            Nếu $f$ là một hàm với miền xác định là $D$, đồ thị của nó là một tập hợp các cặp số $$ \{(x,f(x))\vert x\in D\}$$
        \end{tcolorbox}
    \end{frame}
    %%Hàm thông dụng
    \begin{frame}{Các hàm thông dụng}
        
    \end{frame}