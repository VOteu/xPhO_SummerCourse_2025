\begin{frame}
\frametitle{Tốc Độ}
Tốc độ trung bình: 
\[\overline{v}=\frac{\Delta s}{\Delta t}.\]
Đo tốc độ:
\begin{itemize}
    \item Quãng đường 
    \item Thời gian 
    \item Sai số
\end{itemize}
\vspace{8pt}

Sai số của phép đo ứng với: \(1000\text{m}\rightarrow 1\text{m}\rightarrow 1\text{cm}.\)

Ví dụ: Thời gian đi hết 1cm của một người đang chạy.
\end{frame}
\begin{frame}
\frametitle{Vai trò của giải tích}
     Sự cần thiết của một đại lượng:
     \begin{itemize}
        \item Có thể tính được (dựa trên mô hình toán học)
        \item Thuần tuý toán học (không bị ràng buộc bởi thực nghiệm)
        \item Phản ảnh quy luật chuyển động của vật
     \end{itemize}
     \vspace{8pt}

     \(\implies \text{Tốc độ tức thời và Giải tích}\)
\end{frame}