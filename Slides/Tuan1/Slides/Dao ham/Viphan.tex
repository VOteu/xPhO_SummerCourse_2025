\begin{frame}{Xấp xỉ tuyến tính và vi phân}
    Các giá trị của hàm $y=f(x)$ tại các điểm gần $P(a, f(a))$ rất gần với giá trị của hàm $y=f(a)+f'(a)(x-a)$ là tiếp tuyến của đường cong $y=f(x)$ tại điểm $P$.
    
    \textit{(Hình vẽ)}
\end{frame}
\begin{frame}{Xấp xỉ tuyến tính và vi phân}
    \begin{tcolorbox}[colback=blue!10, colframe=blue!50!black, title=Định nghĩa]
    Phép tính xấp xỉ
    \begin{equation}
        f(x)\approx f(a)+f'(a)(x-a)
    \end{equation}
    được gọi là \textbf{xấp xỉ tuyến tính}.

    Hàm tuyến tính mà đồ thị của nó là tiếp tuyến này
    \begin{equation}
        L(x)=f(a)+f'(a)(x-a)
    \end{equation}
    được gọi là \textbf{tuyến tính hóa} của $f$ tại $a$.
    \end{tcolorbox}
\end{frame}
\begin{frame}{Xấp xỉ tuyến tính và vi phân}
    Khi $x$ càng tiến lại gần $a$:
    \begin{equation}
        f'(x)\Delta x\longrightarrow f(x)-f(a) \equiv \Delta y
    \end{equation}
    \begin{tcolorbox}[colback=blue!10, colframe=blue!50!black, title=Định nghĩa]
    Khi $\Delta x\rightarrow0$, ta kí hiệu $\Delta x=dx$, lúc này, $f'(x)dx$ được gọi là \textbf{vi phân} của hàm $f$, kí hiệu:
    \begin{equation}
        dy=f'(x)dx
    \end{equation}
    \end{tcolorbox}{}
\end{frame}
\begin{frame}{Xấp xỉ tuyến tính và vi phân}
    \begin{tcolorbox}[colback=blue!10, colframe=blue!50!black, title=Định lý]
    Vi phân của hàm số phụ thuộc vào cách ta chọn biến độc lập:
    \begin{equation}
        df(g(x))=(f\circ g)'dx=f'(g)g'(x)dx=f'(g)dg
    \end{equation}
    \end{tcolorbox}
\end{frame}