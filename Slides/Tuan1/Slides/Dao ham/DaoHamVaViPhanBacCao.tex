\begin{frame}{Đạo hàm và vi phân bậc cao}
    \begin{tcolorbox}[colback=blue!10, colframe=blue!50!black, title=Đạo hàm bậc cao]
    Nếu $f$ là hàm khả vi, $f'$ cũng là hàm khả vi, vậy có thể có đạo hàm của $f'$, được gọi là \textbf{đạo hàm bậc hai} của $f$, kí hiệu là $f''$. Quá trình tương tự có thể tiếp diễn: \textbf{Đạo hàm bậc $n$} thường được kí hiệu là $f^{(n)}$ và thu được bằng cách lấy đạo hàm của $f$ $n$ lần.
    \end{tcolorbox}
    \end{frame}
    \begin{frame}{Đạo hàm và vi phân bậc cao}
    \begin{tcolorbox}[colback=blue!10, colframe=blue!50!black, title=Vi phân bậc cao]
    Nếu $f$ là hàm khả vi, $f'$ cũng là hàm khả vi, vậy $f'(x)dx$ có thể có vi phân của nó, được gọi là \textbf{vi phân bậc hai} của $f$, kí hiệu bằng $d^2 f$. Quá trình tương tự có thể tiếp diễn: \textbf{vi phân bậc $n$}, thường được kí hiệu là $d^nf$ và thu được bằng cách lấy vi phân của $f$ $n$ lần, $d^nf=f^{(n)}df^n$.
    \end{tcolorbox}
     \begin{tcolorbox}[colback=blue!10, colframe=blue!50!black, title=Công thức Leibniz]
     $\cdots$
     \end{tcolorbox}
\end{frame}

\begin{frame}{Xấp xỉ tuyến tính của một số hàm thông dụng}
    
\end{frame}