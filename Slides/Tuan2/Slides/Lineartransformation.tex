\begin{frame}
    \frametitle{Hệ phương trình tuyến tính}
    Một hệ hai phương trình tuyến tính ở dạng tổng quát:
\begin{align*}
a_{11}x_{1} + a_{12}x_{2} &= b_{1}, \\
a_{21}x_{1} + a_{22}x_{2} &= b_{2}.
\end{align*}
Tương đương với,
\[\begin{bmatrix}
    a_{11}\\a_{21}
\end{bmatrix}x_{1}+\begin{bmatrix}
    a_{12}\\a_{22}
\end{bmatrix}x_{2}=\begin{bmatrix}
    b_{1}\\b_{2}
\end{bmatrix};\]
\[\begin{bmatrix}
    a_{11}&a_{12}\\
    a_{21}&a_{22}
\end{bmatrix}\begin{bmatrix}
    x_{1}\\x_{2}
\end{bmatrix}=\begin{bmatrix}
    b_{1}\\b_{2}
\end{bmatrix}.\]
Hệ phương trình tuyến tính được gói gọn thành 
\[\mathbf{A}\mathbf{x}=\mathbf{b}.\]
\end{frame}
\begin{frame}
    \frametitle{Nghiệm của hệ phương trình tuyến tính}
    Nghiệm của hệ phương trình tuyến tính đó ở dạng tổng quát:
    \begin{align*}
x_{1} &= a_{11}^{-1}b_{1} + a_{12}^{-1}b_{2}, \\
x_{2} &= a_{21}^{-1}b_{1} + a_{22}^{-1}b_{2}.
\end{align*}
Viết lại trong ngôn ngữ của vector:
\[\begin{bmatrix}
    x_{1}\\x_{2}
\end{bmatrix}=\begin{bmatrix}
    a_{11}^{-1}&a_{12}^{-1}\\
    a_{21}^{-1}&a_{22}^{-1}
\end{bmatrix}\begin{bmatrix}
    b_{1}\\b_{2}
\end{bmatrix}.\]
\[\implies \mathbf{x}=\mathbf{A}^{-1}\mathbf{b}\]
\end{frame}
\begin{frame}
    \frametitle{Ma trận nghịch đảo}
    \begin{tcolorbox}[colback=blue!10, colframe=blue!50!black, title=Định nghĩa]
        \(\mathbf{A}^{-1}\) được gọi là ma trận nghịch đảo của \(\mathbf{A}\), và thoả mãn
        \[\mathbf{A}\mathbf{A}^{-1}=\mathbf{A}^{-1}\mathbf{A}=\mathbf{I},\] với \(\mathbf{I}\) là ma trận đơn vị, thoả mãn \(\mathbf{I}\mathbf{x}=\mathbf{x}\).
    \end{tcolorbox}
Ví dụ cho \(\mathbf{I}\):
\[\begin{bmatrix}
    1&0\\
    0&1
\end{bmatrix},\quad
\begin{bmatrix}
    1&0&0\\
    0&1&0\\
    0&0&1
\end{bmatrix}.\]
\end{frame}
